\documentclass[a4paper,12pt]{article}

%% Language and font encodings
\usepackage[english]{babel}
\usepackage[utf8x]{inputenc}
\usepackage[T1]{fontenc}

%% Sets page size and margins
\usepackage[a4paper,top=3cm,bottom=2cm,left=3cm,right=3cm,marginparwidth=1.75cm]{geometry}

%% Useful packages
\usepackage{amsmath}
\newcommand{\norm}[1]{\left\lVert#1\right\rVert}
\usepackage{amssymb} 
\usepackage{graphicx}
\usepackage[colorinlistoftodos]{todonotes}
\usepackage[colorlinks=true, allcolors=blue]{hyperref}
\usepackage{dsfont}

%% Manual Operators 
\DeclareMathOperator*{\argmin}{arg\,min}
\DeclareMathOperator*{\argmax}{arg\,max}

\title{MCMC Summary}
\author{Karl Jiang}


\begin{document}
\maketitle 


\section{Useful Properties of MCMC} 

The MCMC sampling schemes will some the following properties of MCMC estimators. 


\subsubsection{Invariance} 

For the transition operator $\mathcal{T}$ and target density functi$\pi$, we would hope that 

$$
\lim_{k \to \infty} \mu \mathcal{T}^k = \pi
$$ 

So a natural property for the chain when constructing $q(y | x)$ would be invariance - that is $\pi\mathcal{T}(x) = \pi(x)$. This ensures stabillity of the sampling density at step k. 

\subsubsection{Irreducibllity} 

The chain is called irreducible if for all $X^{(k)}$ on every set $B$ that has positive value in $\pi(x)$: 

$$
P_x[X^{(m)} \in B ] > 0 \textrm{ for some } m \in \mathbb{N}
$$

Note that irreducibillity and Invariance aren't enough - we must also show that the process can reach every region in the space of $\pi$, but it does so infinitly often (?) 

\subsubsection{Detailed Balance} 

Often, to satisy the invariance condition, the chain will also satisfy detailed balance: 

$$
P_x[X^{(1)} \in B_1 \textrm{ and } X^{(0)} \in B_0] = 
P_x[X^{(1)} \in B_0 \textrm{ and } X^{(0)} \in B_1] 
$$

Which then allows

$$
\begin{aligned} 
	P_{\pi}[X^{(1)} \in B] &= 
	P_{\pi}[X^{(1)} \in B \textrm{ and } X^{(0)} \in B ] +
	P_{\pi}[X^{(1)} \in B \textrm{ and } X^{(0)} \notin B ] \\
	&= 
	P_{\pi}[X^{(1)} \in B \textrm{ and } X^{(0)} \in B ] +
	P_{\pi}[X^{(1)} \notin B \textrm{ and } X^{(0)} \in B ] \\
	&= 
	P_{\pi}[X^{(1)} \in B \textrm{ and } X^{(0)} \in B ] +
	P_{\pi}[X^{(1)} \notin B \textrm{ and } X^{(0)} \in B ] \\
	&= 
	P_{\pi}[X^{(0)} \in B] 
\end{aligned} 
$$

This insures invariance of chain. 

$\\$
Another way to express the detailed balance condition is 
$$
p(y|x)\pi(x) = p(x|y)\pi(y) 
$$
And in terms of expectations: 
$$
\int g(x)[\mathcal{T}f](x)\pi(x)dx = \int [\mathcal{T}g](x)f(x)\pi(x)dx 
$$

By using the $B_1 = B^c = \notin B$ for the detailed balance definition at step 2.  


\end{document}


